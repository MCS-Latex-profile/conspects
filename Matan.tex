\documentclass{article}
\usepackage{graphicx} % Required for inserting images
\usepackage[T2A]{fontenc}
\usepackage{amsmath}
\usepackage{graphicx}%Вставка картинок правильная
\usepackage{float}%"Плавающие" картинки
\usepackage{wrapfig}%Обтекание фигур (таблиц, картинок и прочего)
\usepackage{epstopdf}
\usepackage{caption} %заголовки плавающих объектов
\usepackage[unicode, pdftex]{hyperref}
\usepackage[usenames]{color}
\usepackage{colortbl}
\usepackage[utf8]{inputenc}
\usepackage[english,russian]{babel}
\usepackage{amsfonts}
\usepackage[all]{xy}
\usepackage{tikz}
\usepackage{amssymb}
\usepackage{blindtext}
\usepackage{hyperref}
\usepackage{subfig}
\usetikzlibrary{}
\newcommand{\eqdef}{\stackrel{\mathrm{def}}{=}}
\captionsetup[figure]{name=Рис}
\title{Теория МКН СПБГУ}

\begin{document}

\begin{titlepage}
\begin{center}
{\LARGE \textbf{Теория с лекций. МатАн}}
\end{center}
\begin{figure}[h]
\centering
\end{figure}
\end{titlepage}

\begin{titlepage}
\clearpage
\textcolor{blue}{\tableofcontents}
\end{titlepage}

\section{Мат. Анализ}
\subsection{Описание $\mathbb{R}$ чисел}
\begin{enumerate}
\item $a+(b+c) = (a+b)+c$
\item $a+0=0+a=a$
\item $a+(-a)=0$
\item $a+b=b+a$
\item $\mathbb{R} \ 0$ - абелева группа
\item $(ab)c=a(bc)$
\item $aa^{-1}=1$
\item $a*1=1*a=a$
\item $ab=ba$\\
Теоремка. Докажем, что $0>1$\\
Предположим, что $0 \geq 1 \Leftrightarrow -1 \geq 0 \Leftrightarrow (-1)(-1) \geq 0 \Leftrightarrow 1 \geq 0$\\
Акиома Архимеда.\\
$x \in \mathbb{R}, \exists n \in \mathbb{N}:x \leq n$\\
Следствие: $x \geq 0 \wedge x \leq \frac{1}{n} \Rightarrow \forall n \in \mathbb{N}: x=0 $\\
\end{enumerate}
Аксиома полноты (Кантора-Дедекинда)\\
$X,Y -$ непустые$, \in \mathbb{R}:\\
\forall x \in X, y \in Y, x \leq y \Rightarrow \exists c \in \mathbb{R}: x \leq c \leq y$.\\
Это порождает, например, $\sqrt{2}$\\
X - ограничено сверху, если $\exists x_o, \forall x \in X: x \leq x_0$. Аналогично определяется ограничение снизу. $x_0$ - верхняя грань.\\
Наименьшая верхняя грань - супремум. Наибольшая меньшая грань - инфинум.
\subsection{Теорема о вложенных промежутках}
$(a;b)$ - интервал или промежуток. По определению:\\
$\forall c \in (a,b): a<c<b$ - для строго так же (лень вводить)\\
Лемма о вложенных промежутках.\\
Пусть $I_1,I_2,...; I_i = [a_i;b_i] \\
I_{i+1} \in I_i$\\
Тогда существует такое c, что $c \in \cap I$\\
\subsection{Функция и последовательность}
\begin{equation*}
|x| = \begin{cases}
x, x \geq 0 \\
-x, x <0
\end{cases}
\end{equation*}
$\forall x \in X: f(x) \in Y \\
x,y -$числа в $\mathbb{R}\\
f - $функция$\\
$Если Y - числа,то f - функционал\\
$\lbrace x_n \rbrace_{k=1}^{\infty } $ - числовая последовательность.\\
Будем говорить, что послед имеет предел в точке a:\\
$\forall \epsilon >0; \exists N: \forall n > N: |x_n-a|<\varepsilon$\\
Обозначаем как: $\lim\limits_{x \to x_0} x_n = a$\\
Определени предпоследовательности и предела на бесконечность - очевидна.\\
Теорема.\\
Если $x_n$ - ограничена сверху и снизу, то она сходится.\\
Теорема Веерштрасса о монотонной последовательности.\\
$x_n \uparrow, \forall n, \exists c; x \leq c; \forall n \Rightarrow \exists \lim\limits_{x\to \infty} x_n$
\subsection{Пределы послеодовательности, разные определения}
$\forall \varepsilon >0: \exists x \in X; x \neq x_0: 0<|x-x_0|<\varepsilon\\
\forall \varepsilon >0; \exists \delta = \delta (\varepsilon):0<|x-x_0|<\delta; |f(x)-a|<\varepsilon$\\
Предел по Коши (первое определение непонятно к чему тут, просто лектор его дал, смирись)\\
Теперь предел по Гейне:\\
$\lim\limits_{x\to x_0} f(x) =a \eqdef \forall x_n \in X; \lim\limits_{x \to \infty} x_n = x_0: f(x_n) = a$\\
Доказательство равносильности определений:

\end{document}