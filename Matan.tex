\documentclass{article}
\usepackage{graphicx} % Required for inserting images
\usepackage[T2A]{fontenc}
\usepackage{amsmath}
\usepackage{graphicx}%Вставка картинок правильная
\usepackage{float}%"Плавающие" картинки
\usepackage{wrapfig}%Обтекание фигур (таблиц, картинок и прочего)
\usepackage{epstopdf}
\usepackage{caption} %заголовки плавающих объектов
\usepackage[unicode, pdftex]{hyperref}
\usepackage[usenames]{color}
\usepackage{colortbl}
\usepackage[utf8]{inputenc}
\usepackage[english,russian]{babel}
\usepackage{amsfonts}
\usepackage[all]{xy}
\usepackage{tikz}
\usepackage{amssymb}
\usepackage{blindtext}
\usepackage{hyperref}
\usepackage{subfig}
\usetikzlibrary{}
\newcommand{\eqdef}{\stackrel{\mathrm{def}}{=}}
\captionsetup[figure]{name=Рис}
\title{Теория МКН СПБГУ}

\begin{document}

\begin{titlepage}
\begin{center}
{\LARGE \textbf{Теория с лекций. МатАн}}
\end{center}
\begin{figure}[h]
\centering
\end{figure}
\end{titlepage}

\begin{titlepage}
\clearpage
\textcolor{blue}{\tableofcontents}
\end{titlepage}

\section{Мат. Анализ}
\subsection{Описание $\mathbb{R}$ чисел}
\begin{enumerate}
\item $a+(b+c) = (a+b)+c$
\item $a+0=0+a=a$
\item $a+(-a)=0$
\item $a+b=b+a$
\item $\mathbb{R} \ 0$ - абелева группа
\item $(ab)c=a(bc)$
\item $aa^{-1}=1$
\item $a*1=1*a=a$
\item $ab=ba$\\
Теоремка. Докажем, что $0>1$\\
Предположим, что $0 \geq 1 \Leftrightarrow -1 \geq 0 \Leftrightarrow (-1)(-1) \geq 0 \Leftrightarrow 1 \geq 0$\\
Акиома Архимеда.\\
$x \in \mathbb{R}, \exists n \in \mathbb{N}:x \leq n$\\
Следствие: $x \geq 0 \wedge x \leq \frac{1}{n} \Rightarrow \forall n \in \mathbb{N}: x=0 $\\
\end{enumerate}
Аксиома полноты (Кантора-Дедекинда)\\
$X,Y -$ непустые$, \in \mathbb{R}:\\
\forall x \in X, y \in Y, x \leq y \Rightarrow \exists c \in \mathbb{R}: x \leq c \leq y$.\\
Это порождает, например, $\sqrt{2}$\\
X - ограничено сверху, если $\exists x_o, \forall x \in X: x \leq x_0$. Аналогично определяется ограничение снизу. $x_0$ - верхняя грань.\\
Наименьшая верхняя грань - супремум. Наибольшая меньшая грань - инфинум.
\subsection{Теорема о вложенных промежутках}
$(a;b)$ - интервал или промежуток. По определению:\\
$\forall c \in (a,b): a<c<b$ - для строго так же (лень вводить)\\
Лемма о вложенных промежутках.\\
Пусть $I_1,I_2,...; I_i = [a_i;b_i] \\
I_{i+1} \in I_i$\\
Тогда существует такое c, что $c \in \cap I$\\
\subsection{Функция и последовательность}
\begin{equation*}
|x| = \begin{cases}
x, x \geq 0 \\
-x, x <0
\end{cases}
\end{equation*}
$\forall x \in X: f(x) \in Y \\
x,y -$числа в $\mathbb{R}\\
f - $функция$\\
$Если Y - числа,то f - функционал\\
$\lbrace x_n \rbrace_{k=1}^{\infty } $ - числовая последовательность.\\
Будем говорить, что послед имеет предел в точке a:\\
$\forall \epsilon >0; \exists N: \forall n > N: |x_n-a|<\varepsilon$\\
Обозначаем как: $\lim\limits_{x \to x_0} x_n = a$\\
Определени предпоследовательности и предела на бесконечность - очевидна.\\
Теорема.\\
Если $x_n$ - ограничена сверху и снизу, то она сходится.\\
Теорема Веерштрасса о монотонной последовательности.\\
$x_n \uparrow, \forall n, \exists c; x \leq c; \forall n \Rightarrow \exists \lim\limits_{x\to \infty} x_n$
\subsection{Пределы послеодовательности, разные определения}
$\forall \varepsilon >0: \exists x \in X; x \neq x_0: 0<|x-x_0|<\varepsilon\\
\forall \varepsilon >0; \exists \delta = \delta (\varepsilon):0<|x-x_0|<\delta; |f(x)-a|<\varepsilon$\\
Предел по Коши (первое определение непонятно к чему тут, просто лектор его дал, смирись)\\
Теперь предел по Гейне:\\
$\lim\limits_{x\to x_0} f(x) =a \eqdef \forall x_n \in X; \lim\limits_{x \to \infty} x_n = x_0: f(x_n) = a$\\
Доказательство равносильности определений:\\
\textbf{(to be continued...)}\\
\subsection{Свойства пределов и непрерывность.}
Опр непрерывности в точке $x_0:\\
x, x_0 \in X: \exists \lim\limits_{x \to x_0} f(x) = f(x_0)$\\
\textbf{Арифметические действия с пределами:}\\
\begin{enumerate}
\item $\lim\limits_{x \to x_0} f(x) \pm \lim\limits_{x \to x_0} g(x) =\lim\limits_{x \to x_0} f(x) \pm g(x)$
\item $\lim\limits_{x \to x_0} f(x)* \lim\limits_{x \to x_0} g(x) =\lim\limits_{x \to x_0} f(x)*g(x)$
\item $ \frac{\lim\limits_{x \to x_0} f(x)}{\lim\limits_{x \to x_0} g(x)}  =\lim\limits_{x \to x_0} \frac{f(x)}{g(x)}; g(x_0) \neq 0 $
\end{enumerate}
\textbf{Доказательство первого}:\\
$\lim\limits_{x \to x_0} f(x) = a \Leftrightarrow \forall \varepsilon >0; \exists \delta_f >0: 0<|x-x_0|<\delta : |f(x)-a|<\varepsilon $\\
Аналогично для $g(x)$. Выберем $\delta = min(\delta_f (\frac{\varepsilon}{2}) ;\delta_g(\frac{\varepsilon}{2}))$.\\
Тогда будет справедливо следующее соотношение:\\
$|f(x)+g(x)-a-b|\leq |f(x)-a|+|g(x)-b|<2*\frac{\varepsilon}{2}=\varepsilon$\\
Аналогично с двумя другими свойствами.\\
Односторонний предел в точке $x_0$ определяется как:\\ 
$\exists \lim\limits_{x \to x_0-} f(x) = a \Leftrightarrow \forall \varepsilon >0: \exists \delta >0: x_0\delta < x< x_0: |f(x)-a|<\varepsilon$\\
Пределльный переход в неравенствах:\\
$g(x) \leq f(x) \Leftrightarrow b \leq a$\\
Доказательство:\\
Пусть $ b>a: min(\delta_f(\frac{b-a}{2}); \delta_g(\frac{b-a}{2}) = \delta: 0<|x-x_0|<\delta:\\ |f(x)-a|<\frac{b-a}{2}; |f(x)-b|< \frac{b-a}{2}$
Получили противоречие.\\
\textbf{Теорема о двух копах:}\\
$h,g,f \in D_f: f(x) \leq g(x) \leq h(x)$\\
Если  $\exists \lim\limits_{x \to x_0} f(x) = \lim\limits_{x \to x_0} h(x) = a \Leftarrow \exists \lim\limits_{x \to x_0} g(x) = a$\\
Доказательство:\\
$\delta = min(\delta_1(\varepsilon);\delta_2(\varepsilon)): |x-x_0|<\delta: f(x),g(x) \in (a-\varepsilon;a+\varepsilon) \Rightarrow g(x) \in (x_0-\varepsilon; x_0+\varepsilon)$\\
Аналогично теорема о копах для последовательностей.\\
\subsection{Сходимость в себе}
Опр. $\lbrace x_n \rbrace_{n=1}^{\infty}: \forall \varepsilon >0; \exists N: \forall n,m > N: |x-x_m|<\varepsilon$\\
\textbf{Теорема.}\\
$\lbrace x_n \rbrace_{n=1}^{\infty}$ - сходится$\Leftrightarrow \lbrace x`_n \rbrace_{n=1}^{\infty}$
Доказательство:
Прямое утверждение очевидно.\\
Докажем обратное: $\varepsilon = \frac{1}{2^n}, n\in \mathbb{N}: |x_n-x_m|<\frac{1}{2^n}$.\\
Зафиксируем n. Тогда любые точки лежат $(x_n-\frac{1}{2^n};x_n+\frac{1}{2^n}) - I_n\\
|I_n| \to 0, n \to \infty$\\
Добъемся вложения интервалов $J_k = \cap I_i; J_k \neq \oslash \Rightarrow J_k$ - интервал. По построению $J_k+1 \in J_k. \exists a \in J_i \forall i: a$-кандидат на предел. (воспользовавшись аксиомой полноты)\\
$\forall \varepsilon >0; \exists N \forall n > N: |x_n-a|<\varepsilon$\\
Аналогично можно определить сходимость функций в себе.\\
\textbf{Теорема}\\
f(x) - сходится в себе $\Leftrightarrow \lim\limits_{x \to x_0} f(x)$\\
Далее рассмотрим гармонический ряд.\\ Доказательство его расходимости очевидное.\\
Аналогично рассмотрим $\xi (n)$ и получим, что при n>1 она сходится. Доказательство тривиально.\\
\subsection{Непрерывность функций}
Непрерывность функции мы уже определяли, определим теперь разрыв I и II рода.\\
Разрыв I рода:\\
$\exists \lim\limits_{x \to -x_0} f(x) \neq \lim\limits_{x \to x_0+} f(x) = a$\\
\textbf{Опр.} $x_0$ - устранимая особенность $\lim\limits_{x \to x_0} f(x) = f(x_0)$\\
Разрыв II рода. $\lim\limits_{x \to -x_0} f(x) \neq f(x_0);\lim\limits_{x \to -x_0} f(x) \neq \lim\limits_{x \to x_0+} f(x); a \neq  \lim\limits_{x \to x_0+} f(x)$\\
\textbf{Свойство непрерывности:}\\
f, g - непрерывны на $E_{f,g}$
\begin{enumerate}
\item $\alpha f + g\beta$ - непрерывны там же
\item $fg$ - непрерывны там же
\item $g(x)\neq 0 \forall x \in D_f: \frac{f}{g}$- непрерывны
\item $f \o g(x)$ -непрерывны
\end{enumerate}
Док-во ласт утверждения:\\
$x_0$ - предельная точка $g(x); f$ - непрерывна в $g(x_0) \Rightarrow f(g(x))$- непрерывна\\
$\forall \varepsilon >0: \exists \delta: 0<|x-x_0|<\delta: |f(g(x))-f(g(x_0))|< \varepsilon\\
\forall \varepsilon >0: \exists \delta_1: 0<|g(x)-g(x_0)|<\delta_1:|f(y)-f(g(x_0))|<\varepsilon\\
\forall \varepsilon >0: \exists \delta_2: 0<|x-x_0|<
\delta_2: |g(x)-g(x_0)|<\varepsilon$\\
\textbf{Мини-следствия:}\\
1) $P(x) = \Sigma a_ix^i$ - непрерывна\\
2) $R(x) = \frac{P(x)}{Q(x)}; Q(x) \neq 0$ - непрерывна\\
Свойства:\\
\textbf{Теорема Веерштрасса (I)}\\
f(x) - непрерывна на [a,b], значит f - ограничена.\\
Доказательство.\\
Пусть f(x) - неограничена на промежутке. Тогда:\\
$\forall n \exists x_n: f(x_n)>n\\
x_{n_k} \to x_0 \in [a,b];\lim\limits_{x \to x_0} f(x) = f(x_0) = \lim\limits_{k \to \infty} f(x_k) = \infty$\\
\textbf{Теорема Веерштрасса (II)}\\
Пусть B непрерывна на [a,b], тогда она достигает своего максимума и минимума, т.е.:
$\exists x_0: f(x_0) \geq f(x), \forall x \in [a,b]\\
\exists y_0: f(y_0) \leq f(x), \forall y \in [a,b]$\\
Посмотри на образ x=[a,b]- оно ограничено.\\
Если X - ограничена сверху $\Rightarrow \exists x_0: Sup(X)=x_0$\\
Если $x_0 \in X \Rightarrow \exists y_0: y_0 = f(x_0)$\\
Если нет, то $\exists y_n \to y_0, y_n \in Y; y_n = f(x_n): \exists x_n: \lbrace x_n \rbrace^{\infty}, x_{n_k} \to x_0$\\
По непрерывности: $f(x_{n_k}) \to f(x_0) = y_0 | \Rightarrow y_0 \in Y$\\
\textbf{Теорема}\\
f(x) непрерывна на [a,b] если $\forall c \in [f(a),f(b)]; \exists x_0:f(x_0) = c; f(a) \leq f(b)$\\
Осуществляем бин поиск точки c. Откуда и получаем непрерывность.\\
Теорема.\\
f - непрерывна на [a,b] и f - инъективна, значит либо f(x)возрастает, либо убывает.\\
Посмотрим на f(a) и f(b), $a \neq b \Rightarrow a>b \wedge a<b.$\\
\begin{equation}
\begin{cases}
f(x_1)>f(x_2)\\
x_1<x_2
\end{cases}
\end{equation}
Значит, если она вначале убывает, потом возрастает, то будет два значения, чуть больших минимального, что противоречит инъективности.\\
\textbf{Лемма}\\
Построим Канторову лестницу. Зададим функции следующим образом. f(0) = 0; f(1) = 1. Будем делить отрезок на 3 равные части, на средней из них, она равна среднему арифметическому от значений на концах отрезка. Таким образом получим непрерывную возрастающую функциию.\\
Можем задать ее так же следующим образом. В средний промежуток попадают числа, которые в троичной системе имееют единицу в разряде.\\
\subsection{Производная}
Введем понятие $o$ и $O$\\
$\lim\limits_{x \to x_0} \frac{f(x)}{g(x)} = 0$\\
$f(x) = O(g(x)); x \to x_0: \exists c: |f(x)|\leq c|g(x)| в x \in \delta$ окрестности.\\
$f(x) = f(x_0)+o(1)$ - непрерывна.\\
$f(x) = A+Bx+o(x-x_0) x \to x_0;\\
A+Bx = f(x_0)+D(x-x_0); f(x) = f(x_0)+D(x-x_0)+o(x-x_0)$\\
Если $\exists \lim\limits_{x \to x_0} \frac{f(x)-f(x_0)}{x-x_0} = D; D = f`$;
$f(x) = f(x_0)+f`(x_0)(x-x_0)+o(x-x_0)$\\
\textbf{Св-ва производной:}\\
\begin{enumerate}
\item $(\alpha f(x)+ \beta g(x) = \alpha f`(x) + \beta g`(x)$
\item $(f(x)*g(x))`=f`(x)*g(x)+g`(x)*f(x)$
\item $(\frac{f(x)}{g(x)})`=\frac{f`(x)*g(x)-g`(x)*f(x)}{g^2(x)}$
\item $f(g(x))`=f`(g(x))*g`(x)$
\end{enumerate}
\end{document}