\documentclass{article}
\usepackage{graphicx} % Required for inserting images
\usepackage[T2A]{fontenc}
\usepackage{amsmath}
\usepackage{graphicx}%Вставка картинок правильная
\usepackage{float}%"Плавающие" картинки
\usepackage{wrapfig}%Обтекание фигур (таблиц, картинок и прочего)
\usepackage{epstopdf}
\usepackage{caption} %заголовки плавающих объектов
\usepackage[unicode, pdftex]{hyperref}
\usepackage[usenames]{color}
\usepackage{colortbl}
\usepackage[utf8]{inputenc}
\usepackage[english,russian]{babel}
\usepackage{amsfonts}
\usepackage[all]{xy}
\usepackage{tikz}
\usepackage{amssymb}
\usepackage{blindtext}
\usepackage{hyperref}
\usepackage{subfig}
\usetikzlibrary{}
\newcommand{\eqdef}{\stackrel{\mathrm{def}}{=}}
\captionsetup[figure]{name=Рис}
\title{Теория МКН СПБГУ}

\begin{document}

\begin{titlepage}
\begin{center}
{\LARGE \textbf{Теория с лекций. Алгебра}}
\end{center}
\begin{figure}[h]
\centering
\end{figure}
\end{titlepage}

\begin{titlepage}
\clearpage
\textcolor{blue}{\tableofcontents}
\end{titlepage}

\section{Алгебра}
\subsection*{Рекомендуемая литература}
 \begin{enumerate}
 \item Кострикин "Основа алгебры" (1том)
 \item Винберг "Курс алгебры"
 \item Лэнг "Алгебра"
 \item Сборник задач по алгебре под редакцией Кострикина
 \item Сборник задач Фадеева и Соминского
 \end{enumerate}
  
\subsection{Кольцо}
Пусть A - некоторое непустое множество. Пусть на нем даны операции: $(a,b) \rightarrow (a+b); (a,b) \rightarrow (ab); a,b \in A$\\
Дадим определение операциям сложения и умножения:\\
\begin{enumerate}
\item $\forall a,b,c \in A : a+(b+c) =(a+b)+c$ - ассоциативность по сложению
\item $\exists 0 \in A, \forall a \in A: a+0=0+a=a$
\item $\forall a \in A, \exists b \in A: a+b = b+a = 0$ b - противополжный a элемент\\
Удостоверимся в том, что 0 - единственный: 
\begin{equation*}
\begin{cases}
a+0=o+a=a\\
a+0`=0`+a = a\\ \forall a \in A
\end {cases}
\leftrightarrow 
 \begin{cases}
0`+0=o+0`=0`\\
0+0`=0`+0 = 0\\ \forall a \in A
\end {cases}
\leftrightarrow 
0=0`
\end{equation*}
\item $\forall a,b \in A: a+b=b+a$ - коммутативность по сложению
\item $\forall a,b,c \in A: (a+b)c=ac+bc=bc+ac$
\end{enumerate}
Множество, которое удовлетворяет этим аксиомам - кольцо в широком смысле. В нашел курсе мы будем рассматривать более частный вариант кольца, добавив еще несколько определений:\\
\begin{enumerate}
\item $\forall a,b \in A: ab=bc$ - коммутативность по умножению
\item $\forall a,b,c \in A: a(bc) = (ab)c$ - ассоциативность по умножению
\item $\forall a \in A: a*1=1*a=a$
\end{enumerate}
Есть кольцо где выполняется еще одна аксиома:\\
$\forall a \in A, a \neq 0, \exists b \in A: ab=ba=1$\\
Если выполнены все эти аксиомы, то такое множество называется полем.\\
Примеры колец:
\begin{enumerate}
\item $\mathbb{Z}$
\item A = 0
\item $d \in \mathbb{N}, \sqrt{d} \neq \mathbb{Z}$, образующие множества $[\sqrt{b}] \mathbb{Z}$
\end{enumerate}
Примеры полей:
\begin{enumerate}
\item $\mathbb{Q}$
\item $\mathbb{R}$
\item $\mathbb{C}$
\end{enumerate}
Если есть 2 кольца A,B, то мы можем образовать их произведение и сложение следующим образом:
\begin{enumerate}
\item $(a_1;b_1)+(a_2;b_2) = (a_1+a_2;b_1+b_2)$
\item $(a_1;b_1)*(a_2;b_2) = (a_1*a_2;b_1*b_2)$
\end{enumerate}
Мы можем аналогично задать произведение $A_1*A_2....*A_n$:\\
Доказательство этого утверждения проведем по индукции (в данном случае свойства ассоциативности):\\
База: при n =3 - доказано по аксиоме.\\
Переход: Есть произведение $(x_1...x_k)(x_{k-1}...x_n)$.\\
Введем понятие левонормированного произведения: $(...(x_1x_2)x_3).....x_n)$ - левонормированное произведение. Такое произведение можно раскрыть с помощью ассоциативности умножения (и привести любое произведение к такому виду).\\
Тогда если в левой скобки будет 1 элемент, то мы получили еще одно левонормированное проивзедение и победили.\\
Если же нет, то тогда рассмотрим две скобки. К правой применим свойство ассоциативности. Тогда вновь получим случай для n-1, т.е мы победили.\\
Аналогично для сложения.\\
Докажем еще одно свойство:
$(-a)b=-ab$\\
Прибавим к обеим частям ab:\\
$ab+(-a)b=ab+(-ab) \leftrightarrow b(a+(-a))=$
\subsection{Понятние идеала для кольца.}
Пусть I - непустое множество A, при этом $x,y \in I \rightarrow x+y \in I; x \in I, a \in A: ax \in I$\\
Если $1 \in I \Rightarrow \forall a \in A: a \in I$\\
Такой идеал называется единичным.\\
0 - всегда принадлежит иделу, докажем это.\\
$x \in I \Rightarrow x*(-1)=-x \in I; x+(-x)=0 \in I$\\
{0} - нулевой идеал \\
Найдем все иделы целых чисел. Возьмем  $k \in \mathbb{Z}, k\mathbb{Z}$ - это идеал. Причем других идеалов нет.\\
Теорема.\\
Для любого идела кольца $\mathbb{Z}$ имеет вид $k\mathbb{Z}, k \in \mathbb{Z}$. Если идеал нулевой, то теорема доказана. Теперь предположим, что идеал не нулевой. Тогда в нем точно будут положительные числа. Пусть k - наименьшее положительное число в I. Тогда $k\mathbb{Z} \in I$.\\
Возьмем $x \in \mathbb{Z}$. Докажем, что $x \vdots k$.\\
Поделим x на k с остатком. $x = kq+r \Rightarrow r = x-kq ; r \in I$. Но r меньше k, противоречие! Так что r = 0.\\
Теорема доказана\\
Ненулевое кольцо - область целостности/область/целостное кольцо, если $\forall a,b \neq 0: ab \neq 0$\\
Если $ab = 0 \Rightarrow a=0 \vee b=0$\\
Приме. $\mathbb{Z}$, любое поле.\\
Область целостности, в котором $\forall$ идеал главный называется кольцом главных идеалов. Идеал главный, если он имеет вид $(a), a \in A$.
Поля можно охарактеризовать как кольца имеющие 2 главных идела (0) и (1).
Фактор кольцо по классу.\\
А - кольцо, I - идел в А.\\
A/I - фактор кольцо по иделу I.\\
$x \in A, x+I = \lbrace x+i, i \in I \rbrace$ - класс эл-та x относ I\\
$\lbrace x+I, x \in A \rbrace$ - мн-во всех классовю.\\
Пример. 
Любые два класса либо не пересекаются, либо совпадают.\\
$c \in (x+I)\cup (y+I); c = x+i_1=y+i_2, i_3 = x-y=i_1-i_2 \in I$
Пусть $C_1, C_2$ - два класса. Докажем, что:\\ $C_1+C_2=(x_1+x_2)+I = (y_1+y_2)+I; \\C_1 = x_1+I=y_1+I; C_2 = x_2+I=y_2+I \\ (y_1+y_2)-(x_1+x_2) = (y_1-x_1)+(y_2-x_2) \in I$\\
$C_1C_2 =x_1x_2+I=y_1y_2+I;\\
y_1y_2-x_1x_2 \in I; y_1y_2-x_1x_2=y_1(y_2-x_2)-y_2(y_1-x_2) \in I$\\
$\mathbb{Z}/n\mathbb{Z} = \lbrace 0+n\mathbb{Z},... (n-1)+n\mathbb{Z} \rbrace, n$ классов,$ n \in \mathbb{Z}, n \neq 0, n>0$\\
\textbf{Изоморфизм колец.}\\
Гомоморфизм колец - пусть A и B - 2 кольца, f:$A \rightarrow B$\\
f - гомоморфизм колец, если выполнены следующие равенства:
\begin{enumerate}
\item $\forall a_1,a_2: f(a_1+a_2) = f(a_1)+f(a_2)$
\item $\forall a_1,a_2: f(a_1a_2) = f(a_1)f(a_2)$
\item $f(1_a) = 1_b$
\end{enumerate}$
f(0)=f(0)+f(0) \Leftrightarrow f(0) = 0$\\
$f(1)=f(1*1) = f(1)*f(1)=1 \Rightarrow f(1) = 1$\\
Пример когда это свойство не выполнено: $\mathbb{Z} \rightarrow \mathbb{Z} \times \mathbb{Z}$\\
Мономорфизм, это гомоморфизм: $\forall a_1,a_2 \in A: f(a_1)=f(a_2) \Rightarrow a_1=a_2$\\
Элиморфизм: $\forall b \in B; \exists a \in A: f(a) = b$\\
Изоморфизм - аналог биекции.\\
\textbf{Операции над множествами.}\\
1) Пересечение: $I_i$ - идеал в А, $\cap I_i$ - идеал в А.\\
2) Произведение конечного числа идеалов - есть идеал.\\
3) Сумма конечного числа идеалов - есть идеал.\\
$\Sigma I_i = \lbrace \Sigma (\alpha_1+\alpha_2+...+\alpha_n) \rbrace$, где $\forall \alpha_i \in I_j$\\
Сумма идеалов - это наименьший по включению идеал, содержащий все идеалы $I_i$.\\
Произведение:$\sqcap I_i = \lbrace \Sigma x_1x_2...x_k; x_i \in I_i \rbrace$\\
Идеалы I и J - взаимнопросты, если: I+J =(1)\\
Пример: A - целые числа, и $(9), (8): 1 =(-1)*8+9*1$
\subsection{Китайская теорема об остатках (КТО)}
А - кольцо, $n>2, I_1,I_2,...I_n$ - идеалы кольца, такие что:\\ $\forall i,j i\neq j:I_i+I_j=(1)$\\
Тогда существует изоморфизм колец $a/\cap I_i  = (A/I_1)\times (A/I_2)\times ...$\\
Устроенный следующим образом:\\
Класс элемента $a`(mod \cup I_i) = a`(mod I_1)a`(mod I_2)...$\\
Проверим, что такое отображение взаимнооднозначно:\\
1) Корректность: если классы одинаковы $a+ \cap I = a` + \cap I \Leftrightarrow a-a` = \cap I \in I$, откуда класс по модулю $I_i$ совпадает с классом по модулю $\cap I_i$.\\
Проверим на гомоморфизм:\\
f - гомоморфизм. $a` (mod \cap I_i), b` (mod \cap I_i): f(a`+b`)=f(a`)+f(b`)$.\\
По определению отношения: $f(a`+b`) = f(a`+b` (mod I_1),...a`+b`(mod I_n))\\
f(a`)+f(b`) = (a`(mod I_1),a`(mod I_2),... + b`(mod I_1),b`(mod I_2)....)$ - что то же самое.\\
Аналогично для проивзведения и единичного элемента.\\
Проверим, что если $a \rightarrow a`, b \rightarrow b`, a`=b` \Rightarrow a=b$\\
Откуда наш гомоморфизм - мономорфизм.\\
$(a`(mod I_1),a`(mod I_2)....) = (b`(mod I_1), b`(mod I_2),...)\\ A \rightarrow A/I, a\rightarrow a`(mod I)$\\ 
Равенство покомпонентно, откуда получим:\\
$a-b \in I_1,I_2,... \Leftrightarrow a-b \in \cap I_i \Rightarrow a` (mod \cap I_i) = b` (mod \cap I_i)$\\
Теперь начнем доказательство КТО по индукции:\\
n=2: $I_1+I_2 = (1), A/I_1 \cap I_2 \simeq A/I_1 \times A/I_2$
Берём любую пару из п.ч.: $b` (mod I_1); c` (mod I_2)\\ 
a \in A: a-b \in I_1, a-c \in I_2\\
a=b+i_1=c+i_2 \Leftrightarrow b-c = i_2-i_1; i_1 \in I_1, i_2 \in I_2\\
I_1+I_2 = (1); x_1+x_2 = 1 \Leftrightarrow (b-c)(x_1+x_2) = (b-c); x_1 \in I_1; x_2 \in I_2\\
x_1(b-c)+x_2(b-c)=(b-c)$\\
Тогда выберем $i_1 = x_1(b-c); i_2=x_2(b-c)$\\
Лемма.\\
Пусть $I_1,I_2...I_n$ - идеалы, такие что $I_1+I_i = (1), \forall i \geq 2$
Тогда $I_1+ \cap I_i = (1)$\\
Доказательство:\\
$x_2,x_3....x_n \in I_1, y_i \in I_i: x_i + y_i = 1\\
\sqcap (x_i+y_i)= y_2y_3...y_n \in \cap I_i+ (x_2*y_1... + .... x_ny_2...) \in I_1 = 1$\\
Откуда по лемме: $A/ \cap I_i \simeq A/I_i \times A/\cap I_i$
Идуктивный переход очевиден.
\subsection{Понятие простого и максимального идеала}
Простой идеал - это такой идеал, что: $I \neq (1); \forall x,y: xy \in I \Rightarrow x\in I \vee y\in I$\\
Максимальный идеал - это такой идеал, что: $I \neq (1); I \in J; I\neq J \Rightarrow J = (1)$\\
Опишем их в конце целых чисел.\\
Теорема.\\
$\forall I_{max} -$ простой.
Доказательство.\\
$x,y \notin I \Rightarrow xy \notin I\\
I \subsetneq I + (x) \Rightarrow I+(x)=(1); I+(y)=(1)$\\
$i_1+ax=1\\
i_2+by=1, a,b \in A\\
i_1i_2+byi_1+axi_2+abxy=1 (\alpha)$\\
Если $xy \in I$. Но тогда $\alpha \in I \Rightarrow 1 \in I \Rightarrow I =(1)$\\
Противоречие, значит $xy \notin I$\\
Если $A = \mathbb{Z}; (a)$ - максимальные идеалы. Если $a \neq 0$ - то идеал не максимальный. $a \neq 1$ - не максимальный.\\
$a \geq 2$
\begin{enumerate}
\item Если a - составное, то a = bc, где b,c$\neq$1. Значит идеал не максимальный $(a)\varsubsetneq (b)$
\item a - простое; a = p. Тогда идеал максимальный. $(p)\subset I \subset \mathbb{Z}$
\end{enumerate}
Чтобы доказать, что он максимальный, нужно доказать, что он не совпадает с $x \in I; x \neq kp;\\
(p) \subset (p)+(x); (x)+(p) = (b) \Rightarrow p \vdots b; x \vdots b \Rightarrow b =p \wedge b=1$\\
Если b=1, то доказано. Если b=p: $(x) \vdots p$- что противоречит условию.\\
Предложение.\\
$I \subset A\\
1) I$ - простое $\Leftrightarrow A/I$ - целостное.
2) I - макс $A/I$ - поле.
\subsection{Евклидовы кольца}
A - целостное, $\exists f(A\lbrace 0 \rbrace )\Rightarrow \mathbb{Z}_0^+;\\
a \rightarrow ||a||; \forall a,b \in A, b\neq 0: \exists q,r \in A: a = bq + r, r =0 \vee r\neq 0; ||r||<||b||$\\
Пример.\\
A - кольцо целых чисел, ||b||=|b|.\\
Иногда в опр включают ||ab||$\geq$||a||\\
Теорема.\\
$\forall$ евклидова кольца - является кольцом главных идеалов.\\
Доказательство:\\
Пусть I - ненулевой идеал; $a \in I:\\
||a|| = min ||i||; i\in I`\lbrace \rbrace$\\
Возьмем $i \in I; i = aq+r$ r - остаток при делении а на b.\\ 
$r \neq 0: r = i-aq \in I; ||r||\leq ||a||$ - противоречие.\\
$I \subset (a) \Rightarrow I = (a)$\\
Пример.\\
$A = \frac{a+b\sqrt{-19}}{2}; a,b \in \mathbb{Z}; a \equiv b (mod 2)$\\
Кольцо главных идеалов, но не евкидово.\\
\subsection{Факториальное кольцо}
A - область целостности. a - обратимый, если $\exists b \in A: ab=1; a \in A$ - неразложимый элемент, если:\\
1) a - необратимый\\
2) $a=bc \Leftarrow b \vee a$ - обратимые.\\
(a) - простой, тогда a - неразлодимый элемент.\\
$a=bc \in (a) \Rightarrow b \in (a) \vee c \in (a)\\
b = ad \Rightarrow a=adc; a(1-dc)=0; cd = 1 \Rightarrow c - $обратимо\\
Опр.\\
Область целостности A - факториальное кольцо, если $\forall a \neq 0 \in A:$\\
1) Существует разложение $a=up_1p_2..., p_i$  -неразложимые, u - обратимое.\\
2) Если существует 2 разложения: $a=up_1p_2...p_n=vq_1q_2...q_m, $ тогда n=m; Существует взаимнооднозначное отображение $(1,2,...,n)\rightarrow (1,2,...m)$\\
Теорема.\\
Любое кольцо главных идеалов - факториальное.\\
$(x,y) \subset k[x,y]\\
\exists$ разложение$0 \neq a; a=up_1p_2...p_n$
Доказательство.\\
Предположим для элемента а такого разложения не существует.\\
$(a) \subset (a_1) \subset ... \subset (a_n), a_i \in A$\\
---\\
Лемма.\\
Такая цепочка со временем стабилизируется, т.е. начиная с некоторого i все они совпадают.\\
$\exists n: (a_n)=(a_{n+1})=...$\\
Доказательство.\\
$\cup (a_k)$ - идеал.\\
$x,y \in \cup (a_i)$, будем считать, что они принадлежит одному идеалу.\\
$x \in (a_k);y \in (a_s), k\in s \Rightarrow x,y \in (a_i) \Rightarrow x+y \in (a_i)$\\
Любой идеал главный $\Rightarrow \cup (a_i) = (b) \Rightarrow b \in (a_k) \Rightarrow (b) \subset (a_k) \subset (b) \Rightarrow (b)=(a_k)$\\
Получили, что: $(a) \subset (a_1) \subset.... \subset(a_k)=(a_{k+1})=... $\\
---\\
Вернемся к док-ву.\\
a - неразложимый, значит он представим в виде bc, где b,c -необратимы, и один из них неразложим.\\
Пусть b - неразложим.\\
Сравним главные идеалы a и b: $(a) \varsubsetneq (b)$. Иначе получили бы a=adc, откуда получили бы, что c - обратимый.\\
Построим идеал строго больший a. Тогда мы аналогично смогли бы построить $(a)\varsubsetneq (a_1)\varsubsetneq ...$, что противоречит лемме.\\
Докажем, что разложение единственно.\\
$a=up_1p_2...p_n=vq_1q_2...q_m$\\
Проведем индукцию по количеству элементов в правой части.\\
База: m=0 - очевидно.\\
m>0: $q_m \Rightarrow up_1p_2...p_n \in (q_m)$\\
Докажем, что $(p_m)$ - простой.\\
Пусть $\pi$ неразложимый элемент элемент $\Rightarrow (\pi)$- простой.\\
Проверим: пусть $xy\in (\pi); (\pi;x);x\notin \pi = (a)\\
\pi \vdots a \Rightarrow \pi = ab, \pi -$ неразложимый$\Rightarrow a \vee b$ - обратимы.\\
Если b - обратимо, то $(\pi) = (a) \Rightarrow (\pi, x) = (a)$ противоречие\\
Если a - обратиый, то $(a)=(1) \Rightarrow (\pi,x)=(1)\\
\alpha \pi + \beta x =1 /*y \Leftrightarrow \alpha y \pi + \beta xy=y \Rightarrow y\in \pi$\\
Что и требовалось$\Rightarrow p_n \in (q_m)\\
p_n = q_m*u_m$.\\
Получаем резуьтат (Евклидовы кольца)$\subset$(Кольца главных идеалов)$\subset$(факториальные кольца)
\subsection{Разложение простых на сумму квадратов}
Задача. Выяснить какие $p = a^2+b^2$ и в каких случаях оно единственно?\\
Предложение.\\
Кольцо $\mathbb{Z}[i] = \lbrace a+bi\rbrace, где a,b \in \mathbb{Z}; i = \sqrt{-1}$, евклидово относительно $||u+iv||=u^2+v^2$\\
$a+bi;c+di\neq 0 \in \mathbb{Z}[i];\\
(a+bi)\neq k(c+di), k\in \mathbb{Z}$\\
Рассмотрим комплексное число:\\
$\frac{a+bi}{c+di}=\alpha + \beta i$. Т.к. $a,b,c,d \in \mathbb{Z} \Rightarrow \alpha, \beta \in \mathbb{Q}$;\\
$|\alpha - |\alpha|\leq 0; |\alpha|$- тут, это окрукгление, т.е.$\alpha \in [m;m+1] \Rightarrow \exists u:|u-\alpha|\leq 0,5$.\\
Аналогично определим v для $\beta$.\\
$a+bi=(\alpha + \beta i)(c+ di)=(c+di)(u+vi)+(c+di)((\alpha-u)+(\beta - u)i)\\
||(c+di)((\alpha-u)+(\beta - v)i))||=||c+di||*||(\alpha-u)+(\beta-v)i||=\\
=(c^2+d^2)((\alpha-u)^2+(\beta-v)^2)\leq (c^2+d^2)*\frac{1}{2}=\frac{||c+di||}{2}<||c+di||$\\
Откуда получаем,что кольцо евклидово, и из вышедоказанного, что оно факториальное.\\
Возьмем произвольное число вида 4k+1=p и посмотрим будет ли оно простым в кольце.\\
p - простое. $p\mathbb{Z}=(p) - max$ идеал\\
$\mathbb{Z}/p\mathbb{Z}$- поле. Докажем это:\\
Вощьмем ненунлевой остаток такой, что при домножении на него получаются ненулевые и попарно различные остатки. То что они не нулевые следует из факториальности.\\
Берем $\overline{k_1a}=k_2\overline{a} \Leftrightarrow \overline{a}(\overline{k_1}-\overline{k_2})\vdots p \Leftarrow k_1-k_2 \vdots p \Leftarrow k_1=k_2$\\
---\\
Докажем теорему Вильсона.\\
p - простое $\Rightarrow (p-1)!+1 \vdots p$\\
Возьмем $p\geq 5 \overline{a} \in \lbrace \overline{2},...,\overline{p-2} \rbrace = P_k\\
\overline{a}-\overline{b}=\overline{1};$ Выкинем такую пару остатков a и b. Проделаем эту итерацию так, что в итоге получится:\\
$P_k = \cup (a_i;b_i) \Rightarrow \sqcap \overline{a_i}=\overline{1}; \overline{p-1} = \overline{-1}\\
\sqcap \overline{a_i} = 1*(-1)=-1 \Rightarrow (p-1)!+1 \vdots p$.\\
---\\
Пусть p = 4k+1; Разделим его остатки следующим образом: $\overline{(1)},.. \overline{(\frac{p-1}{2})},..., -\overline{(\frac{p-1}{2})}$\\
При перемножении получим 1, т.к. их количество четное.  По т.Вильсона:\\
$(\frac{p-1}{2})!+1 \vdots p$\\
Пусть $(\frac{p-1}{2})! = x; x^2+1 \vdots p$\\
Мы нашли x удовлетворяющее условию $(x-i)(x+i) \vdots p \in \mathbb{Z}[i] \Rightarrow p $ неразложимое\\
Пусть:\\
$p=(a+bi)(с+di)\\
p=(a-bi)(c-di)\\
p^2 = (a^2+b^2)(c^2+d^2) \Leftrightarrow (a^2+b^2=p)\vee (c^2+d^2=p)$. Т.к. не один из множетелей не равен 1, иначе p был бы обратимым. Т.е. получили, что и хотели:\\
$p=a^2+b^2=c^2$\\
Проверим единственность представления.\\
$p=(a+bi)(a-bi)=(c+di)(c-di)$ где множители неразложимы.\\
Пусть $a+bi$ раскладывается в 2 необратимых элемента, тогда анологично получаем:\\
$a^2+b^2 = (r^2+s^2)(\overline{r^2}+\overline{s^2})$ - противоречие.\\
Значит, получаем, что: $(a+bi)=(c\pm di)*u, u\in \lbrace \pm 1;\pm i\rbrace$
\end{document}