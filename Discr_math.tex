\documentclass{article}
\usepackage{graphicx} % Required for inserting images
\usepackage[T2A]{fontenc}
\usepackage{amsmath}
\usepackage{graphicx}%Вставка картинок правильная
\usepackage{float}%"Плавающие" картинки
\usepackage{wrapfig}%Обтекание фигур (таблиц, картинок и прочего)
\usepackage{epstopdf}
\usepackage{caption} %заголовки плавающих объектов
\usepackage[unicode, pdftex]{hyperref}
\usepackage[usenames]{color}
\usepackage{colortbl}
\usepackage[utf8]{inputenc}
\usepackage[english,russian]{babel}
\usepackage{amsfonts}
\usepackage[all]{xy}
\usepackage{tikz}
\usepackage{amssymb}
\usepackage{blindtext}
\usepackage{hyperref}
\usepackage{subfig}
\usetikzlibrary{}
\newcommand{\eqdef}{\stackrel{\mathrm{def}}{=}}
\captionsetup[figure]{name=Рис}
\title{Теория МКН СПБГУ}

\begin{document}

\begin{titlepage}
\begin{center}
{\LARGE \textbf{Теория с лекций. Алгебра}}
\end{center}
\begin{figure}[h]
\centering
\end{figure}
\end{titlepage}

\begin{titlepage}
\clearpage
\textcolor{blue}{\tableofcontents}
\end{titlepage}

\section{Дискретная математика}
\begin{center}
\textit{\textbf{Преподаватель: Пузырина Светлана Александровна\\
s.puzynina@gmail.com}}
\end{center}
\subsection{Булевые функции}
\textbf{Булевая функция} - это функция вида: \\
f: $(0;1)^n \rightarrow (0,1)$
Иными словами сопоставляет кажой n-элементной последовательности значение 1 или 0.\\
Всего для n переменных существует $2^{2^n}$ булевых функций.\\
Булевую функцию можно задавать таблицей истинности или же вектором истинности.\\
\textbf{Базис} - некоторое подмножество булевых функций.\\
\textit{\textbf{Определение.}}\\
Формула над базисом F определяется по индукции.\\
База: всякая функция $f \in F$ является формулой над F.\\
Индуктивный переход: если $f(x_1 , x_2... x_n)$ - формула над базисом F, а $\phi_1 , \phi_2... \phi_n$ - либо формулы 
над F, либо переменные, но тогда $f(\phi_1 , \phi_2... \phi_n)$   формула над базисом F.\\
Пример: $(x\bigvee y)\bigwedge(x\bigvee z)$ - формула над базисом $( \bigvee, \bigwedge )$\\
\textit{Введем удобное обозначение:}\\
\begin{equation*}
x^{\sigma} = 
\begin{cases}
x, \sigma = 1\\
\ulcorner x, \sigma =0\\
\end{cases}
\end{equation*}
\textbf{Простой конъюнкцией} называется конъюнкция одной или нескольких переменных (их отрицаний), где при этом каждая переменная встречается лишь один раз.\\
\textbf{Дезъюнктивно нормальная форма (ДНФ)} - представление БФ в виде дизъюнкции простых конъюнкций.\\
Если в каждой конъюнкции учавсвуют все переменные, то это \textbf{С(соверщенная)ДНФ}.\\
\textit{Как построить СДНФ?}\\
В таблице истинности отмечаем все наборы переменных, для которых БФ дает значение 1. Для каждого такого набора $(\sigma_1, ... \sigma_n)$ берем конъюнкцию $(x^{\sigma_1}_1, ...x^{\sigma_n}_n)$. Включаем в СДНФ все полученные конъюнкции:\\ $f(x_1...x_n) = \bigwedge (x^{\sigma_1}_1, ...x^{\sigma_n}_n)$\\
\textit{Теорема.}\\
Для любой БФ (не тождественной 0) существует СДНФ.\\
\textbf{Аналогично определим СКНФ:}\\
Простой дизъюнкцией называется дизъюнкция одной или нескольких переменных (их отриц), причем каждая переменная встречается не более одного раза.\\
КНФ - представление БФ в виде конъюнкции простых дизъюнкции.\\
Если в каждой дизъюнкции все переменные - то это СКНФ.\\
Строится аналогично.\\
Причем вопрос о разлодении всех функций в виде СКНФ до сих пор открыт.\\
\textbf{Многочлен жегалкина.}\\
Сумма по мудулю 2 конъюнкций переменных (допускается слагаемое 1) без повторения слагаемых, а так же константа 0.\\
Общий вид:\\
$f(x_1, x_2..., x_n) = a \bigoplus a_{i_1i_2....i_k}* x_{i1}*...*x_{ik2}$\\
Зачастую константу 0 не считают полиномом Жегалкина, т.е. в полиноме допускаются лишь константа 1 и сложение с дизъюнкцией.\\
\textit{Теорема.}\\	
Для каждой функцией существует единственное представление многочленом Жегалкина.\\
Док-во.\\
\textit{Существование.} Преобразуем ДНФ:\\
Замена дизъюнкции: $x \wedge y = x \oplus y \oplus x \wedge y$\\
Замена отрицаний: $!x = x \oplus 1$\\
Раскрываем скобки: $(x \oplus y) \vee z = (x \vee z) \oplus (x \vee z)$\\
Сокращаем: $x \oplus x = 0$\\
\textit{Единственнность:} всего многочленов $2^{2^n}$, столько же сколько и БФ, а значит такое представление единственное\\
\textbf{Замкнутые классы.}
F - множество БФ замыкание $\llcorner F\lrcorner$ (относительно суперпозиции) - это множество всех булевых функций, представимых формулой над F.\\
\textbf{Замкнутый класс} - это класс равный своему замыканию.\\
\textbf{Некоторые замкнутые классы:}
\begin{enumerate}
\item $T_0$ - класс функций сохраняющих 0.
$f(0...0) = 0$
\item $T_1$ - класс функций сохраняющих 1.
$f(1...1) = 1$
\item Двойственные функции к f:
$f*(x_1.....x_n) = !f(!x_1.....!x_n)$
\item Частичный порядок на множестве двоичных наборов: $(b_1, .... b_n)\leq (c_1,.....c_n)$, если $b_i \leq c_i$.\\
f - монотонная функция, если $f(\alpha) \leq f(\beta)$, если $\alpha \leq \beta$
\item Линейные функции - такие, что их многочлен Жегалкина не использует конъюнкций, а так же константа 0.
\end{enumerate} 

\subsection{Теорема Пост.}
\textbf{Критерий полноты сисемы функций.}\\
Множество БФ F называется полной системой, если все булевые функции выразимы формулами над этим базисом.\\
\textbf{Теорема.}
Множество булевых функций F является полным т и т.т, когда F не содержится ни в одном из пяти классов $T_0, T_1, S, M, L$.
1) Если содержится, то его замыкание также содержится в этом классе.
2) Если не содержится, то  есть функции не из данных классов (аналогично).\\
Пусть $f_0 $
\end{document}