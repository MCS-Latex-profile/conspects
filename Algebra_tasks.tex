\documentclass{article}
\usepackage{graphicx} % Required for inserting images
\usepackage[T2A]{fontenc}
\usepackage{amsmath}
\usepackage{graphicx}%Вставка картинок правильная
\usepackage{float}%"Плавающие" картинки
\usepackage{wrapfig}%Обтекание фигур (таблиц, картинок и прочего)
\usepackage{epstopdf}
\usepackage{caption} %заголовки плавающих объектов
\usepackage[unicode, pdftex]{hyperref}
\usepackage[usenames]{color}
\usepackage{colortbl}
\usepackage[utf8]{inputenc}
\usepackage[english,russian]{babel}
\usepackage{amsfonts}
\usepackage[all]{xy}
\usepackage{tikz}
\usepackage{amssymb}
\usepackage{blindtext}
\usepackage{hyperref}
\usepackage{subfig}
\usetikzlibrary{}
\newcommand{\eqdef}{\stackrel{\mathrm{def}}{=}}
\captionsetup[figure]{name=Рис}
\title{Теория МКН СПБГУ}

\begin{document}

\begin{titlepage}
\begin{center}
{\LARGE \textbf{Задачи к экзамену по Алгебре}}
\end{center}
\begin{figure}[h]
\centering
\end{figure}
\end{titlepage}

\subsection*{Задача 1}
Пусть $p_1,p_2...p_n - $ простые идеалы над A.\\
$I\subset A$ - идеал. Пусть $I \subset \cup p_i$\\
Доказать, что:\\
$\exists i: I \subset p_i$

\subsection*{Задача 2}

Доказать, что I - максимальный, значит $A/I$- поле

\subsection*{Задача 3}

Доказать, что если A - евклидово кольцо, с нормой $|| ||_1$, то на A существует норма:\\
1) А - евклидово относительно $|| ||_2$\\
2)$||ab||_2\geq||b||_2; \forall a,b \neq 0$

\subsection*{Задача 4}

Кольцо $\mathbb{Z}$ - евклидово, при этом $\mathbb{Z}[\sqrt{5}]$ -нет.\\
Доказать, что:\\
1)$\mathbb{Z}[\sqrt{-2}]$ - евклидово\\
2)$A = \lbrace \frac{a+b\sqrt{3}}{2}\rbrace, a,b \in \mathbb{Z}, a\equiv b (mod 2)$ - евклидово
\subsection*{Задача 5}
Доказать теорему Дирихле в случае:\\
1)$a=3;d=4$\\
2)$a=1;d=4$
\subsection*{Задача 6}
1)Доказать, что (3) не простой в $\mathbb{Z}[\sqrt{-5}]$\\
2)Доказать, что (3) раскалывается в виде произведения двух простых идеалов. Найти эти идеалы.
\subsection*{Задача 7}
Док-ать, что $\frac{2+i}{2-i}$ не является степенью единицы
\subsection*{Задача 8}
Доказать, что $(x-1)^m-x^m+1 \vdots (x^2-x+1)^2$

\end{document}